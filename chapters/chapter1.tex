% ============================================================
%  Chapter 1 — Template Documentation
% ============================================================
\chapter{Template Documentation} \label{chap:template}

This chapter is the complete documentation for the NOVA IMS LaTeX Thesis Template.
It explains how every configuration file works, what each option does, and how to
adapt the template for your own thesis.  Read this chapter carefully before you
start writing.  Once you are comfortable, you can delete or archive it and replace
it with your own Introduction.

% ------------------------------------------------------------
\section{Overview} \label{sec:overview}
% ------------------------------------------------------------

The template is a simplified fork of the NOVAthesis project, redesigned for
NOVA IMS doctoral and master programmes.  The key design goals are:

\begin{itemize}
  \item \textbf{Flat structure.}  All your writing lives under four top-level
        directories: \texttt{chapters/}, \texttt{figures/}, \texttt{bibliography/},
        and \texttt{config/}.  There are no deeply nested subdirectories.
  \item \textbf{Single entry-point.}  You compile \texttt{main.tex}.  You should
        never need to touch it.
  \item \textbf{Separation of concerns.}  Metadata, structure, style, and content
        are each controlled by a different file.
  \item \textbf{One-to-one with the Word template.}  Cover pages, fonts, margins,
        heading styles, list-of-figures/tables, and back-matter layout all match
        the official NOVA IMS Word document.
\end{itemize}

% ------------------------------------------------------------
\section{Directory Structure} \label{sec:structure}
% ------------------------------------------------------------

\begin{description}
  \item[\texttt{main.tex}] Compile entry-point.  Loads the style, all
        config files, and calls the loader macros defined in
        \texttt{config/files.tex}.  \emph{Do not edit this file.}

  \item[\texttt{Makefile}] Runs the full compilation chain
        (\texttt{latexmk} + \texttt{biber}) and auto-generates
        \texttt{config/bibliography-sources.tex} from every \texttt{.bib}
        file found in \texttt{bibliography/}.

  \item[\texttt{config/cover.tex}] Thesis metadata: title, author, supervisors,
        degree, submission date.  \emph{This is the first file you should edit.}
        See Section~\ref{sec:cover}.

  \item[\texttt{config/files.tex}] Controls which chapters, appendices, and
        annexes are compiled, and in what order.
        See Section~\ref{sec:files}.

  \item[\texttt{config/bibliography.tex}] Selects the bibliography style
        (\texttt{apa}, \texttt{numeric}, or \texttt{authoryear}).
        See Section~\ref{sec:bibliography}.

  \item[\texttt{config/bibliography-sources.tex}] Auto-generated by
        \texttt{make}.  Lists all \texttt{\textbackslash addbibresource}
        calls.  \emph{Do not edit by hand.}

  \item[\texttt{config/packages.tex}] Add your own \texttt{\textbackslash usepackage}
        calls and custom macros here.
        See Section~\ref{sec:packages}.

  \item[\texttt{chapters/preface.tex}] Front-matter sections: Statement of
        Integrity, Dedication, Acknowledgements, Quote, Abstract (EN + PT),
        Glossary, Acronyms, and Symbols.
        See Section~\ref{sec:preface}.

  \item[\texttt{chapters/chapter*.tex}] Your main thesis chapters, one file each.

  \item[\texttt{chapters/appendix*.tex}] Appendix chapters.  Numbered with Latin
        letters (A, B, \ldots) and labelled ``Appendix''.

  \item[\texttt{chapters/annex*.tex}] Annex chapters.  Numbered with Roman
        numerals (I, II, \ldots) and labelled ``Annex''.

  \item[\texttt{figures/}] All figures used anywhere in the document.
        Reference them with a path relative to the project root,
        e.g.\ \verb|\includegraphics{figures/myplot.pdf}|.

  \item[\texttt{bibliography/*.bib}] Your BibTeX/BibLaTeX source files.
        Drop any number of \texttt{.bib} files here; the \texttt{Makefile}
        picks them all up automatically.

  \item[\texttt{style/novaims-dgi.sty}] The consolidated style file.
        Controls cover rendering, heading formats, page layout, running
        headers, and annex/appendix numbering.
        \emph{Only edit if you know what you are doing.}
\end{description}

% ------------------------------------------------------------
\section{Configuring Your Cover} \label{sec:cover}
% ------------------------------------------------------------

Open \texttt{config/cover.tex}.  Every metadata item is set with a
\texttt{\textbackslash renewcommand}.  The following listing shows all
available options and what they control.

\begin{description}
  \item[\texttt{\textbackslash coverlanguage}]
        Language of the cover page text.  Set to \texttt{en} (English) or
        \texttt{pt} (Portuguese).  This also controls which cover PDF asset is
        used from \texttt{style/assets/covers/DGI/}.

  \item[\texttt{\textbackslash thesistitle}]
        Full title of your thesis, exactly as it should appear on the cover.

  \item[\texttt{\textbackslash thesisauthor}]
        Your full name, as it should appear on the cover and title page.

  \item[\texttt{\textbackslash thesisdegree}]
        The degree being awarded, e.g.\ \texttt{PhD in Information Management}
        or \texttt{Master in Data Science and Advanced Analytics}.

  \item[\texttt{\textbackslash thesissubmissiondate}]
        Month and year of submission, e.g.\ \texttt{January 2026}.

  \item[\texttt{\textbackslash thesisschoolEN}]
        English name of the school.  Defaults to
        \texttt{NOVA Information Management School}.

  \item[\texttt{\textbackslash thesisschoolPT}]
        Portuguese name of the school.

  \item[\texttt{\textbackslash thesisuniversity}]
        University name.  Defaults to \texttt{NOVA University Lisbon}.

  \item[\texttt{\textbackslash thesissupervisors}]
        Primary supervisor(s).  Use \texttt{\textbackslash\textbackslash} to
        separate lines and \texttt{\textbackslash textit\{\}} for the
        affiliation in italics.  Example:
\begin{verbatim}
\renewcommand{\thesissupervisors}{%
  João Silva\\
  \textit{Associate Professor, NOVA IMS}
}
\end{verbatim}

  \item[\texttt{\textbackslash thesiscosupervisors}]
        Co-supervisor(s), using the same format.  Leave the body empty
        (i.e., \texttt{\{\}}) if there is no co-supervisor; the field will
        be suppressed automatically.

  \item[\texttt{\textbackslash printbackcovertrue} / \texttt{\textbackslash printbackcoverfalse}]
        Controls whether the back cover page is printed.  Set to
        \texttt{\textbackslash printbackcoverfalse} for digital-only submission.
\end{description}

% ------------------------------------------------------------
\section{Managing Chapters, Appendices, and Annexes} \label{sec:files}
% ------------------------------------------------------------

\texttt{config/files.tex} is the structural switchboard of your thesis.
It defines several boolean flags and loader macros.

\subsection{Boolean flags} \label{sec:flags}

\begin{description}
  \item[\texttt{\textbackslash perchapterbibliographytrue/false}]
        When \texttt{true}, each chapter prints its own \emph{References}
        section at its end.  When \texttt{false}, a single consolidated
        bibliography is printed at the end of the document.

  \item[\texttt{\textbackslash printlistoffigurestrue/false}]
        Enables or disables the List of Figures in the front matter.

  \item[\texttt{\textbackslash printlistoftablestrue/false}]
        Enables or disables the List of Tables in the front matter.

  \item[\texttt{\textbackslash printappendicestrue/false}]
        Enables or disables the entire appendix back-matter block.

  \item[\texttt{\textbackslash printannexestrue/false}]
        Enables or disables the entire annex back-matter block.
\end{description}

\subsection{Adding a chapter} \label{sec:addchapter}

Open \texttt{config/files.tex} and add a line inside the
\texttt{\textbackslash loadchapters} macro:

\begin{verbatim}
\newcommand{\loadchapters}{%
  \chapterwithrefs{chapters/chapter1}
  \chapterwithrefs{chapters/chapter2}
  \chapterwithrefs{chapters/chapter3}   % <-- new chapter
}
\end{verbatim}

The helper \texttt{\textbackslash chapterwithrefs\{\}} inputs the file and,
when per-chapter bibliographies are enabled, prints the local References
section automatically.

\subsection{Adding an appendix} \label{sec:addappendix}

Add a line inside \texttt{\textbackslash loadappendices}.  Also make sure
\texttt{\textbackslash printappendicestrue} is set:

\begin{verbatim}
\newcommand{\loadappendices}{%
  \ifprintappendices
    \imsstartappendices
    \chapterwithrefs{chapters/appendix1}
    \chapterwithrefs{chapters/appendix2}
    \chapterwithrefs{chapters/appendix3}   % <-- new appendix
  \fi
}
\end{verbatim}

\subsection{Adding an annex} \label{sec:addannex}

Add a line inside \texttt{\textbackslash loadannexes} and set
\texttt{\textbackslash printannexestrue}:

\begin{verbatim}
\newcommand{\loadannexes}{%
  \ifprintannexes
    \imsstartannexes
    \chapterwithrefs{chapters/annex1}
    \chapterwithrefs{chapters/annex2}    % <-- new annex
  \fi
}
\end{verbatim}

\textbf{Important:} The \texttt{\textbackslash imsstartappendices} and
\texttt{\textbackslash imsstartannexes} calls must appear \emph{once}, before
the first appendix/annex file respectively.  They reset the chapter counter
and switch the chapter prefix to ``Appendix'' / ``Annex''.

% ------------------------------------------------------------
\section{Configuring the Bibliography} \label{sec:bibliography}
% ------------------------------------------------------------

Open \texttt{config/bibliography.tex}.  Three \texttt{\textbackslash usepackage\{biblatex\}}
blocks are provided; exactly one must be uncommented at a time.

\begin{description}
  \item[APA (default)] Author-date format per the APA 7th edition.
        Citations look like (Silva, 2024) or Silva (2024).
        Use \texttt{\textbackslash parencite\{\}} for parenthetical and
        \texttt{\textbackslash textcite\{\}} for narrative citations.
        Note that the line \texttt{\textbackslash AtBeginDocument\{\textbackslash let\textbackslash cite\textbackslash parencite\}}
        in \texttt{bibliography.tex} makes \texttt{\textbackslash cite}
        behave like \texttt{\textbackslash parencite} globally.

  \item[Numeric] Citations are numbered in order of appearance: [1], [2], \ldots.
        Bibliography is sorted by citation order.

  \item[Author-year] Similar to APA but without full APA formatting rules
        (less strict journal/book field formatting).
\end{description}

To add a new \texttt{.bib} file, simply place it in \texttt{bibliography/} and
re-run \texttt{make}.  The \texttt{Makefile} regenerates
\texttt{config/bibliography-sources.tex} automatically.

% ------------------------------------------------------------
\section{Adding Custom Packages and Macros} \label{sec:packages}
% ------------------------------------------------------------

Put all your extra \texttt{\textbackslash usepackage} calls and custom command
definitions in \texttt{config/packages.tex}.  This file is loaded after the
style and before the document body, so your definitions are available everywhere.
Never load packages in individual chapter files.

\begin{verbatim}
% config/packages.tex — example
\usepackage{tikz}
\usetikzlibrary{arrows,positioning}

\newcommand{\R}{\mathbb{R}}   % shorthand for real numbers
\newcommand{\E}{\mathbb{E}}   % shorthand for expectation
\end{verbatim}

% ------------------------------------------------------------
\section{Writing the Front Matter} \label{sec:preface}
% ------------------------------------------------------------

All front-matter content lives in \texttt{chapters/preface.tex}.
Each section is introduced with \texttt{\textbackslash prefacesection\{\textit{name}\}}.

\subsection{SDG icons} \label{sec:sdg}

The template prints UN Sustainable Development Goal icons on the abstract page.
Control which icons appear with:

\begin{verbatim}
\setsdglanguage{en}        % en or pt
\setsdgs{3, 11, 13}        % comma-separated list of goal numbers (1-17)
\end{verbatim}

\subsection{Abstract and Resumo} \label{sec:abstract}

The English abstract uses \texttt{\textbackslash prefacesection\{Abstract\}} followed by
your abstract text and a \texttt{\textbackslash keywords\{\}} command:

\begin{verbatim}
\prefacesection{Abstract}
Your abstract text here.
\keywords{keyword one, keyword two, keyword three}
\end{verbatim}

The Portuguese abstract follows immediately with
\texttt{\textbackslash prefacesection\{Resumo\}}.

\subsection{Glossary, Acronyms, and Symbols}

These use the standard \texttt{description} environment inside a
\texttt{\textbackslash prefacesection}:

\begin{verbatim}
\prefacesection{Acronyms}
\begin{description}
  \item[NLP]  Natural Language Processing
  \item[ML]   Machine Learning
\end{description}
\end{verbatim}

% ------------------------------------------------------------
\section{Using the Chapter Front Matter} \label{sec:chapterfrontmatter}
% ------------------------------------------------------------

For article-style chapters (e.g., papers submitted for publication), the template
provides a \texttt{chapterfrontmatter} environment that prints a chapter-level
abstract and keywords immediately after the chapter heading:

\begin{verbatim}
\chapter{My Chapter Title} \label{chap:mychapter}

\begin{chapterfrontmatter}
  \begin{abstract}
    \small \noindent Chapter-level abstract text here.
  \end{abstract}
  \begin{keyword}
    \small
    keyword one \sep keyword two \sep keyword three
  \end{keyword}
\end{chapterfrontmatter}

\section{Introduction}
...
\end{verbatim}

For standard thesis chapters (not article-style), omit the
\texttt{chapterfrontmatter} block and start directly with
\texttt{\textbackslash section\{\}}.

% ------------------------------------------------------------
\section{Compiling the Document} \label{sec:compile}
% ------------------------------------------------------------

\subsection{With make (recommended)} \label{sec:make}

\begin{verbatim}
make
\end{verbatim}

This single command:
\begin{enumerate}
  \item Scans \texttt{bibliography/*.bib} and writes
        \texttt{config/bibliography-sources.tex}.
  \item Runs \texttt{latexmk} with \texttt{-pdf} and \texttt{-shell-escape}.
  \item Calls \texttt{biber} for bibliography processing.
  \item Repeats passes until the output is stable.
\end{enumerate}

The final output is \texttt{main.pdf}.

\subsection{Manual compilation} \label{sec:manual}

If \texttt{make} is not available, run:

\begin{verbatim}
latexmk -shell-escape -file-line-error -pdf main
\end{verbatim}

Or step by step:

\begin{verbatim}
pdflatex -shell-escape main
biber main
pdflatex -shell-escape main
pdflatex -shell-escape main
\end{verbatim}

\subsection{Overleaf} \label{sec:overleaf}

\begin{enumerate}
  \item Download the repository as a ZIP.
  \item Upload the ZIP to Overleaf.
  \item Set \texttt{main.tex} as the root document
        (\emph{Menu} $\rightarrow$ \emph{Main document}).
  \item Set the compiler to \texttt{pdfLaTeX} and the bibliography tool to
        \texttt{Biber}.
  \item Compile.
\end{enumerate}

A paid Overleaf account may be required due to the 20-second compilation limit
on the free plan.

% ------------------------------------------------------------
\section{Adding a New Programme (for contributors)} \label{sec:newprog}
% ------------------------------------------------------------

Currently, only the DGI cover assets are included.  Adding support for another
NOVA IMS programme requires three steps:

\begin{enumerate}
  \item Create the cover PDFs (EN and PT) following the brand guidelines and
        place them in \texttt{style/assets/covers/PROGRAMME\_CODE/DGI-EN/} and
        \texttt{style/assets/covers/PROGRAMME\_CODE/DGI-PT/}.
  \item Add a new cover-rendering block in \texttt{style/novaims-dgi.sty},
        conditional on the programme code.
  \item Update \texttt{config/cover.tex} to expose the new programme option via
        a \texttt{\textbackslash renewcommand}.
\end{enumerate}

Pull requests are very welcome.

