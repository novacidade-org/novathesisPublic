% ============================================================
%  Annex I — Useful LaTeX Command Reference
% ============================================================
\chapter{Useful \LaTeX{} Command Reference} \label{ann:commands}

This annex is a compact command reference for formatting constructs that
appear frequently in theses but are not covered in Chapter~\ref{chap:latex}.

% ------------------------------------------------------------
\section{Text Formatting} \label{ann:textformat}
% ------------------------------------------------------------

\begin{description}
  \item[\texttt{\textbackslash textbf\{text\}}]
        \textbf{Bold text.}

  \item[\texttt{\textbackslash textit\{text\}}]
        \textit{Italic text.}

  \item[\texttt{\textbackslash texttt\{text\}}]
        \texttt{Monospace / typewriter text.}  Use for code, file names, and
        commands.

  \item[\texttt{\textbackslash emph\{text\}}]
        \emph{Contextual emphasis.}  Italic in normal context; roman inside
        an already-italic environment.  Prefer this over
        \texttt{\textbackslash textit} for emphasis.

  \item[\texttt{\textbackslash underline\{text\}}]
        \underline{Underlined text.}  Use sparingly.

  \item[\texttt{\textbackslash small}, \texttt{\textbackslash footnotesize},
        \texttt{\textbackslash large}, \texttt{\textbackslash Large}]
        Relative font-size switches.  These are declarations (not
        commands with arguments), so wrap them in a group:
        \verb|{\small small text}|.

  \item[\texttt{\textbackslash textsc\{text\}}]
        \textsc{Small capitals.}

  \item[\texttt{\textbackslash textsuperscript\{n\}}]
        Superscript: 1\textsuperscript{st}, 2\textsuperscript{nd}.

  \item[\texttt{\textbackslash textsubscript\{n\}}]
        Subscript in text mode.
\end{description}

% ------------------------------------------------------------
\section{Spacing and Breaks} \label{ann:spacing}
% ------------------------------------------------------------

\begin{description}
  \item[\texttt{\textasciitilde}]
        Non-breaking space.  Always use between a word and a
        \texttt{\textbackslash ref} or number:
        \verb|Figure~\ref{fig:x}|, \verb|Table~1|.

  \item[\texttt{\textbackslash,}]
        Thin space.  Useful in maths and units: \verb|$3\,\mathrm{km}$|.

  \item[\texttt{\textbackslash quad}, \texttt{\textbackslash qquad}]
        Wide horizontal spaces (1 em and 2 em respectively).

  \item[\texttt{\textbackslash newline} or \texttt{\textbackslash\textbackslash}]
        Force a line break within a paragraph.  Avoid in normal prose.

  \item[\texttt{\textbackslash newpage}]
        Force a page break.  Avoid until the final formatting pass.

  \item[\texttt{\textbackslash noindent}]
        Suppress paragraph indentation on the following paragraph.

  \item[\texttt{\textbackslash vspace\{1cm\}}]
        Insert vertical space.  Use with care; prefer structural commands.
\end{description}

% ------------------------------------------------------------
\section{Special Characters} \label{ann:special}
% ------------------------------------------------------------

Some characters have special meaning in \LaTeX{} and must be escaped:

\begin{center}
\begin{tabular}{ll}
  \toprule
  Character & Command \\
  \midrule
  \%        & \texttt{\textbackslash\%} \\
  \$        & \texttt{\textbackslash\$} \\
  \&        & \texttt{\textbackslash\&} \\
  \#        & \texttt{\textbackslash\#} \\
  \_        & \texttt{\textbackslash\_} \\
  \{        & \texttt{\textbackslash\{} \\
  \}        & \texttt{\textbackslash\}} \\
  \textasciitilde & \texttt{\textbackslash textasciitilde} \\
  \textasciicircum & \texttt{\textbackslash textasciicircum} \\
  \textbackslash & \texttt{\textbackslash textbackslash} \\
  \bottomrule
\end{tabular}
\end{center}

% ------------------------------------------------------------
\section{Dashes and Quotation Marks} \label{ann:dashes}
% ------------------------------------------------------------

\begin{description}
  \item[Hyphen (\texttt{-})] Word-hyphenation, compound adjectives:
        \emph{well-known}, \emph{data-driven}.

  \item[En-dash (\texttt{-{}-})] Ranges: pages 12--15, years 2020--2026.

  \item[Em-dash (\texttt{-{}-{}-})] Interruption or parenthetical aside---like
        this one---in running text.

  \item[Opening double quotes (\texttt{``})] Typed with two backticks:
        ``quoted text''.

  \item[Closing double quotes (\texttt{''})] Typed with two apostrophes.
\end{description}

Never use the keyboard \texttt{"} character for quotes in \LaTeX.

% ------------------------------------------------------------
\section{Floats and Placement} \label{ann:floats}
% ------------------------------------------------------------

\begin{description}
  \item[\texttt{h}] Place \emph{here} (approximately).
  \item[\texttt{t}] Place at the \emph{top} of a page.
  \item[\texttt{b}] Place at the \emph{bottom} of a page.
  \item[\texttt{p}] Place on a dedicated float \emph{page}.
  \item[\texttt{!}] Override \LaTeX's internal quality thresholds.
\end{description}

The specifier \texttt{[htbp]} is the recommended default.  If \LaTeX{}
still cannot place a float satisfactorily, use
\texttt{\textbackslash FloatBarrier} (from the \texttt{placeins} package)
to force all pending floats to be placed before continuing.

% ------------------------------------------------------------
\section{Common Environments} \label{ann:envs}
% ------------------------------------------------------------

\begin{description}
  \item[\texttt{verbatim}]
        Typesets text exactly as typed, in monospace.  No \LaTeX{} commands
        are interpreted.  Use for code samples and command examples.

  \item[\texttt{quote}]
        Indented single-paragraph quotation.

  \item[\texttt{quotation}]
        Indented multi-paragraph quotation with first-line indentation.

  \item[\texttt{center}]
        Centres content horizontally.

  \item[\texttt{flushleft} / \texttt{flushright}]
        Left- or right-aligns content.

  \item[\texttt{minipage}]
        Creates a boxed region of specified width.  Useful for placing two
        blocks side by side without a figure environment.
\end{description}

