% ============================================================
%  Chapter 2 — Writing in LaTeX: A Practical Guide
% ============================================================
\chapter{Writing in \LaTeX: A Practical Guide} \label{chap:latex}

This chapter is a hands-on guide to the \LaTeX{} constructs you will use most
when writing your thesis.  Rather than showing only code snippets, each section
demonstrates the feature \emph{live} so you can see the rendered output directly
in this document alongside the source that produced it.  The figures, tables,
and citations in this chapter all use real assets from this template repository,
so the document compiles out-of-the-box.

% ------------------------------------------------------------
\section{Document Structure} \label{sec:sections}
% ------------------------------------------------------------

Chapters are divided with the standard sectioning hierarchy.  The template
numbers them automatically:

\begin{verbatim}
\section{My Section}           % e.g. 2.1
\subsection{My Subsection}     % e.g. 2.1.1
\subsubsection{My Sub-sub}     % e.g. 2.1.1.1
\paragraph{Inline heading}     % not numbered
\end{verbatim}

Never skip levels, and avoid more than three levels of nesting in a thesis.

% ------------------------------------------------------------
\section{Labels and Cross-References} \label{sec:labels}
% ------------------------------------------------------------

Attach a \texttt{\textbackslash label\{\}} to any numbered object — chapter,
section, figure, table, equation — and refer to it anywhere with
\texttt{\textbackslash ref\{\}}.  Numbers update automatically.

\subsection{Labelling conventions} \label{sec:labelconv}

Use a consistent prefix to keep labels scannable:

\begin{description}
  \item[\texttt{chap:}]  Chapters      — \verb|\label{chap:introduction}|
  \item[\texttt{sec:}]   Sections      — \verb|\label{sec:methodology}|
  \item[\texttt{fig:}]   Figures       — \verb|\label{fig:accmap}|
  \item[\texttt{tab:}]   Tables        — \verb|\label{tab:summary}|
  \item[\texttt{eq:}]    Equations     — \verb|\label{eq:poisson}|
  \item[\texttt{app:}]   Appendix secs — \verb|\label{app:workflow}|
  \item[\texttt{ann:}]   Annex secs    — \verb|\label{ann:commands}|
\end{description}

\subsection{Making a reference} \label{sec:makeref}

Always put a tilde~(\textasciitilde) between the word and
\texttt{\textbackslash ref} — it is a non-breaking space that prevents
``Figure'' from ending up alone at the end of a line:

\begin{verbatim}
See Section~\ref{sec:methodology} for details.
Figure~\ref{fig:accmap} shows the spatial distribution.
Full results are in Appendix~\ref{app:workflow}.
Raw data are in Annex~\ref{ann:commands}.
Equation~\eqref{eq:poisson} defines the model.   % use \eqref for equations
See page~\pageref{fig:accmap} for the map.
\end{verbatim}

\noindent\textbf{Live example} — this sentence refers back to the subsection
just above: Section~\ref{sec:labelconv} listed the labelling prefixes, and
Figure~\ref{fig:accmap} on page~\pageref{fig:accmap} shows the accident map
introduced in Section~\ref{sec:figures}.

% ------------------------------------------------------------
\section{Figures} \label{sec:figures}
% ------------------------------------------------------------

Place all image files in \texttt{figures/}.  Use a path relative to the
project root in \texttt{\textbackslash includegraphics}.

\subsection{A single figure} \label{sec:singlefig}

The source below produced Figure~\ref{fig:accmap}:

\begin{verbatim}
\begin{figure}[htbp]
  \centering
  \includegraphics[width=0.85\textwidth]{figures/sevAccMap24LAB.png}
  \caption{Spatial distribution of severe accident reports across mainland
           Portugal, 2024.  Darker municipalities had higher daily counts.}
  \label{fig:accmap}
\end{figure}
\end{verbatim}

\begin{figure}[htbp]
  \centering
  \includegraphics[width=0.85\textwidth]{figures/sevAccMap24LAB.png}
  \caption{Spatial distribution of severe accident reports across mainland
           Portugal, 2024.  Darker municipalities had higher daily counts.}
  \label{fig:accmap}
\end{figure}

Prefer \texttt{.pdf}/\texttt{.eps} for vector graphics and \texttt{.png} for
raster images.  The specifier \texttt{[htbp]} tells \LaTeX{} to try: here,
top, bottom, float page.  Avoid hardcoding \texttt{[H]} unless essential.

\subsection{Side-by-side figures} \label{sec:subfigs}

Use the \texttt{subcaption} package (already loaded).
Figure~\ref{fig:twoplots}, using two files already in \texttt{figures/}:

\begin{verbatim}
\begin{figure}[htbp]
  \centering
  \begin{subfigure}{0.48\textwidth}
    \centering
    \includegraphics[width=\textwidth]{figures/sir_day_of_week.png}
    \caption{Reports by day of week.}
    \label{fig:dayofweek}
  \end{subfigure}
  \hfill
  \begin{subfigure}{0.48\textwidth}
    \centering
    \includegraphics[width=\textwidth]{figures/sir_by_month.png}
    \caption{Reports by month.}
    \label{fig:bymonth}
  \end{subfigure}
  \caption{Temporal patterns in severe accident reports, Portugal 2024.}
  \label{fig:twoplots}
\end{figure}
\end{verbatim}

\begin{figure}[htbp]
  \centering
  \begin{subfigure}{0.48\textwidth}
    \centering
    \includegraphics[width=\textwidth]{figures/sir_day_of_week.png}
    \caption{Reports by day of week.}
    \label{fig:dayofweek}
  \end{subfigure}
  \hfill
  \begin{subfigure}{0.48\textwidth}
    \centering
    \includegraphics[width=\textwidth]{figures/sir_by_month.png}
    \caption{Reports by month.}
    \label{fig:bymonth}
  \end{subfigure}
  \caption{Temporal patterns in severe accident reports, Portugal 2024.
           Panel~(a) shows the weekly cycle; panel~(b) shows seasonal
           variation.}
  \label{fig:twoplots}
\end{figure}

Refer to individual panels as Figure~\ref{fig:dayofweek} and
Figure~\ref{fig:bymonth}, or to the pair as Figure~\ref{fig:twoplots}.

% ------------------------------------------------------------
\section{Tables} \label{sec:tables}
% ------------------------------------------------------------

Use \texttt{booktabs} (already loaded): no vertical lines; horizontal rules
only via \texttt{\textbackslash toprule}, \texttt{\textbackslash midrule},
\texttt{\textbackslash bottomrule}.  Captions go \emph{above} tables.
Table~\ref{tab:summary} was produced by the code that follows:

\begin{table}[htbp]
  \centering
  \caption{Descriptive statistics — main variables, mainland Portugal 2024.}
  \label{tab:summary}
  \begin{tabular}{lrrrr}
    \toprule
    Variable                  & Mean  & Std.\ Dev. & Min    & Max   \\
    \midrule
    Severe accidents (daily)  & 3.21  & 2.07       & 0      & 18    \\
    Temperature (°C)          & 17.4  & 6.3        & $-$1.2 & 42.1  \\
    Precipitation (mm/h)      & 0.18  & 0.61       & 0      & 8.4   \\
    Hours above 30°C          & 1.4   & 2.1        & 0      & 11    \\
    Hours of heavy rain       & 0.3   & 0.8        & 0      & 6     \\
    \bottomrule
  \end{tabular}
\end{table}

\begin{verbatim}
\begin{table}[htbp]
  \centering
  \caption{Descriptive statistics.}
  \label{tab:summary}
  \begin{tabular}{lrrrr}
    \toprule
    Variable          & Mean & Std.\ Dev. & Min    & Max  \\
    \midrule
    Accidents (daily) & 3.21 & 2.07       & 0      & 18   \\
    Temperature (°C)  & 17.4 & 6.3        & $-$1.2 & 42.1 \\
    Precipitation     & 0.18 & 0.61       & 0      & 8.4  \\
    \bottomrule
  \end{tabular}
\end{table}
\end{verbatim}

For multi-column headers use \verb|\multicolumn{n}{c}{text}|.  For tables
spanning multiple pages, load \texttt{longtable} in \texttt{config/packages.tex}.

% ------------------------------------------------------------
\section{Mathematical Equations} \label{sec:equations}
% ------------------------------------------------------------

\subsection{Inline mathematics} \label{sec:inlinemath}

Wrap inline maths in dollar signs.  \verb|$\hat{y}_{it} = \mu_i + \varepsilon_{it}$|
renders as $\hat{y}_{it} = \mu_i + \varepsilon_{it}$.

\subsection{Numbered display equations} \label{sec:displayeq}

Use \texttt{equation} and always label equations you will refer to.
The code below produced Equation~\eqref{eq:poisson}:

\begin{verbatim}
\begin{equation}
  \mathbb{E}[y_{it} \mid \mathbf{x}_{it}]
    = \exp\!\left(\alpha_i + \beta_t
      + \mathbf{x}_{it}'\boldsymbol{\gamma}\right)
  \label{eq:poisson}
\end{equation}
\end{verbatim}

\begin{equation}
  \mathbb{E}[y_{it} \mid \mathbf{x}_{it}]
    = \exp\!\left(\alpha_i + \beta_t
      + \mathbf{x}_{it}'\boldsymbol{\gamma}\right)
  \label{eq:poisson}
\end{equation}

Use \verb|\eqref{eq:poisson}| (not \verb|\ref|) — it adds parentheses
automatically so you get ``Equation~\eqref{eq:poisson}''.

\subsection{Multi-line equations} \label{sec:multiline}

\texttt{align} aligns multiple lines at the \texttt{\&} column:

\begin{verbatim}
\begin{align}
  \text{RMSE} &= \sqrt{\frac{1}{n}\sum_{i=1}^{n}(\hat{y}_i - y_i)^2}
               \label{eq:rmse} \\
  \text{MAE}  &= \frac{1}{n}\sum_{i=1}^{n}\lvert\hat{y}_i - y_i\rvert
               \label{eq:mae}
\end{align}
\end{verbatim}

\begin{align}
  \text{RMSE} &= \sqrt{\frac{1}{n}\sum_{i=1}^{n}(\hat{y}_i - y_i)^2}
               \label{eq:rmse} \\
  \text{MAE}  &= \frac{1}{n}\sum_{i=1}^{n}\lvert\hat{y}_i - y_i\rvert
               \label{eq:mae}
\end{align}

Use \texttt{align*} to suppress all equation numbers.

% ------------------------------------------------------------
\section{Citations} \label{sec:citations}
% ------------------------------------------------------------

The template uses \texttt{biblatex} with \texttt{biber} and APA by default.
Drop any number of \texttt{.bib} files into \texttt{bibliography/} and
re-run \texttt{make} — they are all picked up automatically.

\subsection{Citation commands} \label{sec:citecmds}

\begin{description}
  \item[\texttt{\textbackslash cite\{key\}}]
        Parenthetical citation — aliased to \texttt{\textbackslash parencite}
        by this template.  Example source:
        \verb|\cite{andreyWeatherChronicHazard2003}|\\
        Rendered: \cite{andreyWeatherChronicHazard2003}

  \item[\texttt{\textbackslash textcite\{key\}}]
        Narrative citation where the author's name is part of the sentence.
        \verb|\textcite{blackEffectsRainfallVehicle2017}|\\
        Rendered: \textcite{blackEffectsRainfallVehicle2017}

  \item[\texttt{\textbackslash parencite[p.~45]\{key\}}]
        With a page number post-note.\\
        Rendered: \parencite[p.~45]{basaganaHighAmbientTemperatures2015}

  \item[\texttt{\textbackslash parencite\{key1,key2\}}]
        Multiple sources in one bracket.\\
        Rendered: \parencite{andreyWeatherChronicHazard2003,blackEffectsRainfallVehicle2017}

  \item[\texttt{\textbackslash citeauthor\{key\}}]
        Author name only, no year:
        \citeauthor{basaganaHighAmbientTemperatures2015}

  \item[\texttt{\textbackslash citeyear\{key\}}]
        Year only, no author:
        \citeyear{basaganaHighAmbientTemperatures2015}

  \item[\texttt{\textbackslash fullcite\{key\}}]
        Full bibliography entry rendered inline in the text.

  \item[\texttt{\textbackslash nocite\{*\}}]
        Forces all entries in a \texttt{.bib} file into the bibliography,
        whether cited or not.
\end{description}

\subsection{Live paragraph example} \label{sec:livecite}

The paragraph below uses real bib keys from \texttt{bibliography/traffic\_weather.bib}
and renders exactly as it would in your own chapter:

\medskip
\noindent
The relationship between weather and road safety has been widely studied.
\textcite{andreyWeatherChronicHazard2003} report that precipitation increased
crashes by approximately 75\% and injuries by 45\% in mid-sized Canadian cities.
\textcite{basaganaHighAmbientTemperatures2015} document a J-shaped relationship
between temperature and crash risk in Spain.
More recent matched-pair analyses across multiple US states confirm that rainfall
consistently raises collision risk \cite{blackEffectsRainfallVehicle2017}.

% ------------------------------------------------------------
\section{Footnotes} \label{sec:footnotes}
% ------------------------------------------------------------

Insert a footnote inline with \texttt{\textbackslash footnote\{\}}.
The sentence ends, a superscript appears, and the note is set at the foot
of the page.\footnote{This is a live example footnote produced by
\texttt{\textbackslash footnote\{...\}} immediately after the period.}

Keep footnotes short.  If a note exceeds two lines, move the content into
the main text or an appendix.

% ------------------------------------------------------------
\section{URLs and Hyperlinks} \label{sec:urls}
% ------------------------------------------------------------

\texttt{hyperref} is already loaded.  Every \texttt{\textbackslash ref},
\texttt{\textbackslash cite}, and table-of-contents entry is automatically
a clickable link in the PDF.

\begin{description}
  \item[\texttt{\textbackslash url\{...\}}]
        Monospace clickable URL: \url{https://www.novaims.unl.pt}

  \item[\texttt{\textbackslash href\{url\}\{anchor text\}}]
        Hyperlink with custom visible text:
        \href{https://www.novaims.unl.pt}{NOVA IMS official website}

  \item[DOI/URL in bibliography]
        Add \texttt{doi = \{10.xxxx/...\}} or \texttt{url = \{https://...\}}
        to any BibTeX entry and \texttt{biblatex} renders and links it.
\end{description}

You want to use \texttt{href} for most purposes, since if the URL breaks across lines, the link will still work, whereas \texttt{url} can break but only at certain characters.
% ------------------------------------------------------------
\section{Lists} \label{sec:lists}
% ------------------------------------------------------------

All three list environments are shown live below.

\textbf{Bullet list} (\texttt{itemize}):
\begin{itemize}
  \item Crowdsourced incident data (Waze)
  \item Gridded reanalysis weather data (ERA5-Land)
  \item Official accident statistics (ANSR)
\end{itemize}

\textbf{Numbered list} (\texttt{enumerate}):
\begin{enumerate}
  \item Download the repository as a ZIP.
  \item Upload to Overleaf or clone locally.
  \item Edit \texttt{config/cover.tex} with your metadata.
  \item Run \texttt{make} to compile.
\end{enumerate}

\textbf{Definition list} (\texttt{description}):
\begin{description}
  \item[ERA5-Land] ECMWF hourly gridded reanalysis product at 9 km resolution.
  \item[ANSR] Autoridade Nacional de Segurança Rodoviária — the Portuguese road
              safety authority.
  \item[GLM] Generalised Linear Model.
\end{description}

% ------------------------------------------------------------
\section{Keeping the Source Readable} \label{sec:readable}
% ------------------------------------------------------------

\begin{itemize}
  \item Write \textbf{one sentence per line}.  Error messages and Git diffs
        reference line numbers — shorter lines make both much easier to read.
  \item Use \texttt{\%} comments to annotate complex tables and equations.
  \item Never commit generated files (\texttt{*.aux}, \texttt{*.bbl},
        \texttt{main.pdf}) — the \texttt{.gitignore} already excludes them.
  \item When in doubt about a package, run \texttt{texdoc \textit{packagename}}
        in a terminal for the official documentation.
  \item For a version-control workflow, see Appendix~\ref{app:workflow}.
\end{itemize}
